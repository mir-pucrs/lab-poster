\documentclass[a0,portrait]{lab-poster}

% \usepackage[brazil]{babel}    % Configuração de Linguagem (comentar para inglês)
% \renewcommand{\tablename}{Tabela}
% \renewcommand{\figurename}{Figura}

% \newcommand\itemadjust{\itemsep.5em \parskip0pt \parsep0pt}

\title{Your title here}
\author{Author 1, Author 2}

\themecolor{Goldenrod}
\unilogo{\includegraphics[width=8cm]{fig/pucrs-logo.pdf}}
\lablogo{\hspace{-3cm}\includegraphics[width=12cm]{fig/mir-text.pdf}}

\begin{document}
\maketitle

%---------------------------------------------------------------

\begin{multicols}{2} 
%---------------------------------------------------------------
%	MOTIVAÇÃO
%---------------------------------------------------------------
\section*{First Section}

\begin{itemize}
	\item Point 1;
	\item Point 2;
	\item 
\end{itemize}

%---------------------------------------------------------------
%   AGENTES
%---------------------------------------------------------------
\section*{Second Section}

\begin{itemize}
	\item Idea 1; e
	\item Idea 2.
\end{itemize}

%---------------------------------------------------------------
%	AgentSpeak(L)
%---------------------------------------------------------------
\section*{Third Section}

\begin{itemize}
	\item Description 1;
	\item You can see this point in Listing~\ref{alg:environment-example};
	\item Definitions from...
\end{itemize}

\vspace{13mm}


\begin{itemize}
	\item Key ideas...:
	\begin{enumerate}
		% [leftmargin=2em]\itemadjust
		\item Idea 1;
		\item Idea 2; 
		\item Idea 3; e
		\item Idea 4.
	\end{enumerate}	
	\item Figure~\ref{fig:myfig} illustrates the point.
\end{itemize}
\vspace{13mm}

\begin{center}
	%\includegraphics[width=0.99\linewidth]{fig/my-figure.pdf}
	\Huge Figure Here (commented out include)
	\captionof{figure}{Figure example.}
	\label{fig:myfig}
\end{center}	

\columnbreak

\section*{Fourth Section}

\begin{itemize}
	\item More ideas...:
	\begin{enumerate}
		\item Idea 1;
		\item Idea 2; 
		\item Idea 3; e
		\item Idea 4.
	\end{enumerate}	
	\item Figure~\ref{fig:myfig} illustrates this.
\end{itemize}

\begin{minipage}{.235\textwidth}
	\lstset{style=codeStyle}
	\begin{lstlisting}[language=Python, label={alg:environment-example}, caption={Example of an environment definition in Python.}]
	from environment import *
	lt_continue = parse_literal('continue(true)')
	
	class HelloWorldEnv(Environment):
		def __init__(self):
			Environment.__init__(self)
		
		def execute_action(self, agent_name, action):
			self.clear_perceptions()
			getattr(self, action.functor)(list(action.args))
		
		def aloha(self, *args):
			self.add_percept(lt_continue)
			print('Aloha HelloWorldEnv!')
		
		def mahalo(self, *args):
			print('Mahaloing with %s!' % ", ".join(map(str, *args)))
	\end{lstlisting}
\end{minipage}

\section*{Fifth Section}
      
Current Font Sizes:
\begin{itemize}
	\item {\tiny \verb|\tiny| Font Size: \showfontsize}
	\item {\scriptsize \verb|\scriptsize| Font Size: \showfontsize}
	\item {\footnotesize \verb|\footnotesize| Font Size: \showfontsize}
	\item {\small \verb|\small| Font Size: \showfontsize}
	\item {\normalsize \verb|\normalsize| Font Size: \showfontsize}
	\item {\large \verb|\large| Font Size: \showfontsize}
	\item {\Large \verb|\Large| Font Size: \showfontsize}
	\item {\LARGE \verb|\LARGE| Font Size: \showfontsize}
	\item {\huge \verb|\huge| Font Size: \showfontsize}
	\item {\Huge \verb|\Huge| Font Size: \showfontsize}
	\item {\veryHuge \verb|\veryHuge| Font Size: \showfontsize}
	\item {\VeryHuge \verb|\VeryHuge| Font Size: \showfontsize}
	\item {\VERYHuge \verb|\VERYHuge| Font Size: \showfontsize}
\end{itemize}


%---------------------------------------------------------------
%	REFERENCES
%---------------------------------------------------------------
% Uncomment below if you want to use references in your paper.
%\vspace{-10mm}
%\large
%\color{NavyBlue}
%\color{Black}
%\raggedright
%\bibliographystyle{plain}
%\bibliography{poster}

\end{multicols}

%----------------------------------------------------------------------------------------
\end{document}